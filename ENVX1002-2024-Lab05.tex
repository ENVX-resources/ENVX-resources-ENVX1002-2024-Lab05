% Options for packages loaded elsewhere
\PassOptionsToPackage{unicode}{hyperref}
\PassOptionsToPackage{hyphens}{url}
\PassOptionsToPackage{dvipsnames,svgnames,x11names}{xcolor}
%
\documentclass[
  10pt,
  letterpaper,
  DIV=11,
  numbers=noendperiod]{scrartcl}

\usepackage{amsmath,amssymb}
\usepackage{setspace}
\usepackage{iftex}
\ifPDFTeX
  \usepackage[T1]{fontenc}
  \usepackage[utf8]{inputenc}
  \usepackage{textcomp} % provide euro and other symbols
\else % if luatex or xetex
  \usepackage{unicode-math}
  \defaultfontfeatures{Scale=MatchLowercase}
  \defaultfontfeatures[\rmfamily]{Ligatures=TeX,Scale=1}
\fi
\usepackage{lmodern}
\ifPDFTeX\else  
    % xetex/luatex font selection
  \setmonofont[Scale=0.75]{Source Code Pro}
\fi
% Use upquote if available, for straight quotes in verbatim environments
\IfFileExists{upquote.sty}{\usepackage{upquote}}{}
\IfFileExists{microtype.sty}{% use microtype if available
  \usepackage[]{microtype}
  \UseMicrotypeSet[protrusion]{basicmath} % disable protrusion for tt fonts
}{}
\makeatletter
\@ifundefined{KOMAClassName}{% if non-KOMA class
  \IfFileExists{parskip.sty}{%
    \usepackage{parskip}
  }{% else
    \setlength{\parindent}{0pt}
    \setlength{\parskip}{6pt plus 2pt minus 1pt}}
}{% if KOMA class
  \KOMAoptions{parskip=half}}
\makeatother
\usepackage{xcolor}
\setlength{\emergencystretch}{3em} % prevent overfull lines
\setcounter{secnumdepth}{-\maxdimen} % remove section numbering
% Make \paragraph and \subparagraph free-standing
\ifx\paragraph\undefined\else
  \let\oldparagraph\paragraph
  \renewcommand{\paragraph}[1]{\oldparagraph{#1}\mbox{}}
\fi
\ifx\subparagraph\undefined\else
  \let\oldsubparagraph\subparagraph
  \renewcommand{\subparagraph}[1]{\oldsubparagraph{#1}\mbox{}}
\fi

\usepackage{color}
\usepackage{fancyvrb}
\newcommand{\VerbBar}{|}
\newcommand{\VERB}{\Verb[commandchars=\\\{\}]}
\DefineVerbatimEnvironment{Highlighting}{Verbatim}{commandchars=\\\{\}}
% Add ',fontsize=\small' for more characters per line
\usepackage{framed}
\definecolor{shadecolor}{RGB}{241,243,245}
\newenvironment{Shaded}{\begin{snugshade}}{\end{snugshade}}
\newcommand{\AlertTok}[1]{\textcolor[rgb]{0.68,0.00,0.00}{#1}}
\newcommand{\AnnotationTok}[1]{\textcolor[rgb]{0.37,0.37,0.37}{#1}}
\newcommand{\AttributeTok}[1]{\textcolor[rgb]{0.40,0.45,0.13}{#1}}
\newcommand{\BaseNTok}[1]{\textcolor[rgb]{0.68,0.00,0.00}{#1}}
\newcommand{\BuiltInTok}[1]{\textcolor[rgb]{0.00,0.23,0.31}{#1}}
\newcommand{\CharTok}[1]{\textcolor[rgb]{0.13,0.47,0.30}{#1}}
\newcommand{\CommentTok}[1]{\textcolor[rgb]{0.37,0.37,0.37}{#1}}
\newcommand{\CommentVarTok}[1]{\textcolor[rgb]{0.37,0.37,0.37}{\textit{#1}}}
\newcommand{\ConstantTok}[1]{\textcolor[rgb]{0.56,0.35,0.01}{#1}}
\newcommand{\ControlFlowTok}[1]{\textcolor[rgb]{0.00,0.23,0.31}{#1}}
\newcommand{\DataTypeTok}[1]{\textcolor[rgb]{0.68,0.00,0.00}{#1}}
\newcommand{\DecValTok}[1]{\textcolor[rgb]{0.68,0.00,0.00}{#1}}
\newcommand{\DocumentationTok}[1]{\textcolor[rgb]{0.37,0.37,0.37}{\textit{#1}}}
\newcommand{\ErrorTok}[1]{\textcolor[rgb]{0.68,0.00,0.00}{#1}}
\newcommand{\ExtensionTok}[1]{\textcolor[rgb]{0.00,0.23,0.31}{#1}}
\newcommand{\FloatTok}[1]{\textcolor[rgb]{0.68,0.00,0.00}{#1}}
\newcommand{\FunctionTok}[1]{\textcolor[rgb]{0.28,0.35,0.67}{#1}}
\newcommand{\ImportTok}[1]{\textcolor[rgb]{0.00,0.46,0.62}{#1}}
\newcommand{\InformationTok}[1]{\textcolor[rgb]{0.37,0.37,0.37}{#1}}
\newcommand{\KeywordTok}[1]{\textcolor[rgb]{0.00,0.23,0.31}{#1}}
\newcommand{\NormalTok}[1]{\textcolor[rgb]{0.00,0.23,0.31}{#1}}
\newcommand{\OperatorTok}[1]{\textcolor[rgb]{0.37,0.37,0.37}{#1}}
\newcommand{\OtherTok}[1]{\textcolor[rgb]{0.00,0.23,0.31}{#1}}
\newcommand{\PreprocessorTok}[1]{\textcolor[rgb]{0.68,0.00,0.00}{#1}}
\newcommand{\RegionMarkerTok}[1]{\textcolor[rgb]{0.00,0.23,0.31}{#1}}
\newcommand{\SpecialCharTok}[1]{\textcolor[rgb]{0.37,0.37,0.37}{#1}}
\newcommand{\SpecialStringTok}[1]{\textcolor[rgb]{0.13,0.47,0.30}{#1}}
\newcommand{\StringTok}[1]{\textcolor[rgb]{0.13,0.47,0.30}{#1}}
\newcommand{\VariableTok}[1]{\textcolor[rgb]{0.07,0.07,0.07}{#1}}
\newcommand{\VerbatimStringTok}[1]{\textcolor[rgb]{0.13,0.47,0.30}{#1}}
\newcommand{\WarningTok}[1]{\textcolor[rgb]{0.37,0.37,0.37}{\textit{#1}}}

\providecommand{\tightlist}{%
  \setlength{\itemsep}{0pt}\setlength{\parskip}{0pt}}\usepackage{longtable,booktabs,array}
\usepackage{calc} % for calculating minipage widths
% Correct order of tables after \paragraph or \subparagraph
\usepackage{etoolbox}
\makeatletter
\patchcmd\longtable{\par}{\if@noskipsec\mbox{}\fi\par}{}{}
\makeatother
% Allow footnotes in longtable head/foot
\IfFileExists{footnotehyper.sty}{\usepackage{footnotehyper}}{\usepackage{footnote}}
\makesavenoteenv{longtable}
\usepackage{graphicx}
\makeatletter
\def\maxwidth{\ifdim\Gin@nat@width>\linewidth\linewidth\else\Gin@nat@width\fi}
\def\maxheight{\ifdim\Gin@nat@height>\textheight\textheight\else\Gin@nat@height\fi}
\makeatother
% Scale images if necessary, so that they will not overflow the page
% margins by default, and it is still possible to overwrite the defaults
% using explicit options in \includegraphics[width, height, ...]{}
\setkeys{Gin}{width=\maxwidth,height=\maxheight,keepaspectratio}
% Set default figure placement to htbp
\makeatletter
\def\fps@figure{htbp}
\makeatother

% packages
\usepackage{geometry}
\usepackage{xcolor}
\usepackage{eso-pic}
\usepackage{fancyhdr}
\usepackage{sectsty}
\usepackage{fontspec}
\usepackage{titlesec}
% \usepackage{lmodern}
% \usepackage{cmbright}
\usepackage{fvextra}

\DefineVerbatimEnvironment{Highlighting}{Verbatim}{breaklines,commandchars=\\\{\}}
\DefineVerbatimEnvironment{OutputCode}{Verbatim}{breaklines,commandchars=\\\{\}}

% use sans serif font
% \renewcommand{\familydefault}{\sfdefault}

%% colours
\definecolor{primary}{HTML}{FCEDE2}
\definecolor{dark}{HTML}{330033}

%% fontsize
\addtokomafont{subsubsection}{\fontsize{10pt}{12.833pt}\selectfont}



%% Add the border
\AddToShipoutPicture{% 
    \AtPageLowerLeft{% 
        \put(\LenToUnit{\dimexpr\paperwidth-2cm},0){% 
            \color{primary}\rule{2cm}{\LenToUnit\paperheight}%
          }%
     }%
     % logo
    \AtPageLowerLeft{% start the bar at the bottom right of the page
        \put(\LenToUnit{\dimexpr\paperwidth-1.75cm},\paperheight-2.5cm){% move it to the top right
            \color{primary}\includegraphics[width=1.5cm]{assets/usydlogo.png}
          }%
     }%
}

%% Style the page number
\fancypagestyle{usyd}{
  \fancyhf{}
  \renewcommand\headrulewidth{0pt}
  \fancyfoot[R]{\thepage}
  \fancyfootoffset{3.5cm}
}
\setlength{\footskip}{20pt}

%% style the chapter/section fonts
\chapterfont{\color{dark}\fontsize{20}{16.8}\selectfont}
\sectionfont{\color{dark}\fontsize{20}{16.8}\selectfont}
\subsectionfont{\color{dark}\fontsize{14}{16.8}\selectfont}
\titleformat{\subsection}
  {\sffamily\Large\bfseries}{\thesection}{1em}{}[{\titlerule[0.8pt]}]
  
% left align title
\makeatletter
\renewcommand{\maketitle}{\bgroup\setlength{\parindent}{0pt}
\begin{flushleft}
  {\sffamily\huge\textbf{\MakeUppercase{\@title}}} \vspace{0.3cm} \newline
  {\Large {\@subtitle}} \newline
  \@author
\end{flushleft}\egroup
}
\makeatother
\KOMAoption{captions}{tableheading}
\makeatletter
\@ifpackageloaded{tcolorbox}{}{\usepackage[skins,breakable]{tcolorbox}}
\@ifpackageloaded{fontawesome5}{}{\usepackage{fontawesome5}}
\definecolor{quarto-callout-color}{HTML}{909090}
\definecolor{quarto-callout-note-color}{HTML}{0758E5}
\definecolor{quarto-callout-important-color}{HTML}{CC1914}
\definecolor{quarto-callout-warning-color}{HTML}{EB9113}
\definecolor{quarto-callout-tip-color}{HTML}{00A047}
\definecolor{quarto-callout-caution-color}{HTML}{FC5300}
\definecolor{quarto-callout-color-frame}{HTML}{acacac}
\definecolor{quarto-callout-note-color-frame}{HTML}{4582ec}
\definecolor{quarto-callout-important-color-frame}{HTML}{d9534f}
\definecolor{quarto-callout-warning-color-frame}{HTML}{f0ad4e}
\definecolor{quarto-callout-tip-color-frame}{HTML}{02b875}
\definecolor{quarto-callout-caution-color-frame}{HTML}{fd7e14}
\makeatother
\makeatletter
\makeatother
\makeatletter
\makeatother
\makeatletter
\@ifpackageloaded{caption}{}{\usepackage{caption}}
\AtBeginDocument{%
\ifdefined\contentsname
  \renewcommand*\contentsname{Table of contents}
\else
  \newcommand\contentsname{Table of contents}
\fi
\ifdefined\listfigurename
  \renewcommand*\listfigurename{List of Figures}
\else
  \newcommand\listfigurename{List of Figures}
\fi
\ifdefined\listtablename
  \renewcommand*\listtablename{List of Tables}
\else
  \newcommand\listtablename{List of Tables}
\fi
\ifdefined\figurename
  \renewcommand*\figurename{Figure}
\else
  \newcommand\figurename{Figure}
\fi
\ifdefined\tablename
  \renewcommand*\tablename{Table}
\else
  \newcommand\tablename{Table}
\fi
}
\@ifpackageloaded{float}{}{\usepackage{float}}
\floatstyle{ruled}
\@ifundefined{c@chapter}{\newfloat{codelisting}{h}{lop}}{\newfloat{codelisting}{h}{lop}[chapter]}
\floatname{codelisting}{Listing}
\newcommand*\listoflistings{\listof{codelisting}{List of Listings}}
\makeatother
\makeatletter
\@ifpackageloaded{caption}{}{\usepackage{caption}}
\@ifpackageloaded{subcaption}{}{\usepackage{subcaption}}
\makeatother
\makeatletter
\@ifpackageloaded{tcolorbox}{}{\usepackage[skins,breakable]{tcolorbox}}
\makeatother
\makeatletter
\@ifundefined{shadecolor}{\definecolor{shadecolor}{HTML}{e64626}}
\makeatother
\makeatletter
\@ifundefined{codebgcolor}{\definecolor{codebgcolor}{HTML}{F1F1F1}}
\makeatother
\makeatletter
\makeatother
\ifLuaTeX
  \usepackage{selnolig}  % disable illegal ligatures
\fi
\IfFileExists{bookmark.sty}{\usepackage{bookmark}}{\usepackage{hyperref}}
\IfFileExists{xurl.sty}{\usepackage{xurl}}{} % add URL line breaks if available
\urlstyle{same} % disable monospaced font for URLs
\hypersetup{
  pdftitle={Lab 01},
  colorlinks=true,
  linkcolor={blue},
  filecolor={Maroon},
  citecolor={Blue},
  urlcolor={Blue},
  pdfcreator={LaTeX via pandoc}}

\title{Lab 01}
\author{}
\date{Semester 1, 2024}

\begin{document}
\maketitle
\pagestyle{usyd}

\ifdefined\Shaded\renewenvironment{Shaded}{\begin{tcolorbox}[boxrule=0pt, colback={codebgcolor}, borderline west={3pt}{0pt}{shadecolor}, enhanced, frame hidden, sharp corners, breakable]}{\end{tcolorbox}}\fi

\renewcommand*\contentsname{Table of contents}
{
\hypersetup{linkcolor=}
\setcounter{tocdepth}{3}
\tableofcontents
}
\setstretch{1.2}
\hypertarget{welcome}{%
\subsection{Welcome}\label{welcome}}

\begin{tcolorbox}[enhanced jigsaw, rightrule=.15mm, opacityback=0, bottomtitle=1mm, titlerule=0mm, left=2mm, coltitle=black, title=\textcolor{quarto-callout-tip-color}{\faLightbulb}\hspace{0.5em}{Learning outcomes}, breakable, colback=white, toprule=.15mm, colbacktitle=quarto-callout-tip-color!10!white, opacitybacktitle=0.6, arc=.35mm, colframe=quarto-callout-tip-color-frame, bottomrule=.15mm, toptitle=1mm, leftrule=.75mm]

\begin{itemize}
\tightlist
\item
  Learn to use R to calculate a 1-sample t-test
\item
  Apply the steps for hypothesis testing from lectures
\item
  Learn how to interpret statistical output\\
\end{itemize}

\end{tcolorbox}

\hypertarget{before-you-begin}{%
\section{Before you begin}\label{before-you-begin}}

You can download the data

\begin{itemize}
\tightlist
\item
  From module 5 in Canvas
\item
  \href{data/ENVX1002_Data5.xlsx}{ENVX1002\_Data5.xlsx} if you are
  viewing the html file from Github
  \url{https://Github.com/envx-resources}
\end{itemize}

\hypertarget{create-a-new-project}{%
\subsection{Create a new project}\label{create-a-new-project}}

Reminder (skip to step 2 if you are going to use the directory you
created in your tutorial)

\textbf{Step 1:} Create a new project file for the practical put in your
ENVX1002 Folder. \emph{File \textgreater{} New Project \textgreater{}
New Directory \textgreater{} New Project}.

\textbf{Step 2:} Download the data files from canvas or using above link
and copy into your project directory.

I recommend that you make a data folder in your project directory to
keep things tidy! If you make a data folder in your project directory
you will need to indicate this path before the file name.

\textbf{Step 3:} Open a new Quarto file.

i.e.~\emph{File \textgreater{} New File \textgreater{} Quarto Document}
and save it immediately i.e.~File \textgreater{} Save.

\hypertarget{problems-with-your-personal-computer-and-r}{%
\subsection{Problems with your personal computer and
R}\label{problems-with-your-personal-computer-and-r}}

NOTE: If you are having problems with R on your personal computer that
cannot easily be solved by a demonstrator, please use the Lab PCs.

\hypertarget{installing-packages}{%
\subsection{Installing packages}\label{installing-packages}}

All of the functions and datasets in R are organised into packages.
There are the standard (or base) packages which are part of the source
code - the functions and datasets that make up these packages are
automatically available when R is opened. There are also many
contributed packages. These have been written by many different authors,
often to implement methods that are not available in the base packages.
If you are unable to find a method in the base packages, you might be
able to find it in a contributed package. The Comprehensive R Archive
Network (CRAN) site
\href{http://cran.r-project.org/}{(http://cran.r-project.org/)} is where
many contributed packages can be downloaded. Click on packages on the
left hand side. We will download two packages in this class using the
\texttt{install.packages} command and we then load the package into R
using the library command.

Alternatively, in RStudio click on the \emph{Packages tab \textgreater{}
Install \textgreater{} type in package name \textgreater{} click
install}.

\hypertarget{exercise-1-1-sample-t-test-milk-yield---walk-through}{%
\section{Exercise 1: 1-sample t-test Milk Yield - Walk
through}\label{exercise-1-1-sample-t-test-milk-yield---walk-through}}

This exercise will walk you through how to test a hypothesis, check
assumptions and eventually draw a conclusion on your initial hypothesis.
100 cows have their milk yield measured. Suppose we wish to test whether
these milk yields (units unknown) differ significantly from the economic
threshold of 11 units. (The units may possibly be litres of milk
produced on a particular day). The data is in the `Milk.csv' file. You
will follow the steps as outlined in the lectures:

\begin{enumerate}
\def\labelenumi{\arabic{enumi}.}
\tightlist
\item
  Choose level of significance (α)
\item
  Write null and alternate hypotheses
\item
  Check assumptions (normal)
\item
  Calculate test statistic
\item
  Obtain P-value or critical value
\item
  Make statistical conclusion
\item
  Write a scientific (biological) conclusion
\end{enumerate}

Remember you can remember the above using \textbf{HATPC}

Lets go:

\hypertarget{normally-you-choose-0.05-as-a-level-of-significance}{%
\subsection{\texorpdfstring{1. Normally you choose \emph{0.05 as a level
of
significance}:}{1. Normally you choose 0.05 as a level of significance:}}\label{normally-you-choose-0.05-as-a-level-of-significance}}

This value is generally accepted in the scientific community and is also
linked to type 2 errors where choosing a lower significance increases
the likelihood of a type 2 error occurring.

\hypertarget{write-null-and-alternative-hypotheses}{%
\subsection{2. Write null and alternative
hypotheses:}\label{write-null-and-alternative-hypotheses}}

\begin{center}\rule{0.5\linewidth}{0.5pt}\end{center}

\begin{quote}
\textbf{\emph{Question:}} Write down the null hypothesis and alternative
hypotheses:\\
H\textsubscript{0}: \textless{} write your answer here \textgreater{}\\
H\textsubscript{1}: \textless{} write your answer here \textgreater{}
\end{quote}

\begin{center}\rule{0.5\linewidth}{0.5pt}\end{center}

\hypertarget{check-assumptions-normality}{%
\subsection{3. Check assumptions
(normality):}\label{check-assumptions-normality}}

\hypertarget{a.-load-data}{%
\subsubsection{a. load data:}\label{a.-load-data}}

\emph{Make sure you set your working directory first}

\begin{Shaded}
\begin{Highlighting}[]
\CommentTok{\# write your R code here}
\end{Highlighting}
\end{Shaded}

It is always good practice to look at the data first to make sure you
have the correct data, it loaded in correctly and know what the names of
the columns are. This can be done by typing the name of the data
\texttt{Milk} or for large datasets, use \texttt{head()} to show the
first 6 lines:

\begin{Shaded}
\begin{Highlighting}[]
\CommentTok{\# write your R code here}
\end{Highlighting}
\end{Shaded}

\hypertarget{b.-tests-for-normality}{%
\subsubsection{b. Tests for normality:}\label{b.-tests-for-normality}}

qqplots:

\begin{Shaded}
\begin{Highlighting}[]
\CommentTok{\# write your R code here}
\end{Highlighting}
\end{Shaded}

Histogram and boxplots:

\begin{Shaded}
\begin{Highlighting}[]
\CommentTok{\# write your R code here}
\end{Highlighting}
\end{Shaded}

\begin{center}\rule{0.5\linewidth}{0.5pt}\end{center}

\begin{quote}
\textbf{\emph{Question:}} Do the plots indicate the data are normally
distributed?\\
\textbf{\emph{Answer:}} \textless{} write your answer here
\textgreater{}
\end{quote}

\begin{center}\rule{0.5\linewidth}{0.5pt}\end{center}

Shapiro-Wilk test of normality:

\begin{Shaded}
\begin{Highlighting}[]
\CommentTok{\# write your R code here}
\end{Highlighting}
\end{Shaded}

\begin{center}\rule{0.5\linewidth}{0.5pt}\end{center}

\begin{quote}
\textbf{\emph{Question:}} Does the Shapiro-Wilk test indicate the data
are normally distributed? Explain your answer.\\
\textbf{\emph{Answer:}} \textless{} write your answer here
\textgreater{}
\end{quote}

\begin{center}\rule{0.5\linewidth}{0.5pt}\end{center}

\hypertarget{calculate-the-test-statistic}{%
\subsection{4. Calculate the test
statistic}\label{calculate-the-test-statistic}}

In R we achieve this via the command
\texttt{t.test(milk\$Yield,\ mu\ =\ …)} The R output first gives us the
calculated t value, the degrees of freedom, and then the p-value, it
then provides the 95\% CI and the mean of the sample. Were
\texttt{mu\ =\ …} is written enter in the hypothesised mean.

\begin{Shaded}
\begin{Highlighting}[]
\CommentTok{\# write your R code here}
\end{Highlighting}
\end{Shaded}

\hypertarget{obtain-p-value-or-critical-value}{%
\subsubsection{5. Obtain P-value or critical
value}\label{obtain-p-value-or-critical-value}}

\begin{quote}
\textbf{\emph{Question:}} Does the hypothesised economic threshold lie
within the confidence intervals?\\
\textbf{\emph{Answer:}} \textless{} write your answer here
\textgreater{}
\end{quote}

\hypertarget{make-statistical-conclusion}{%
\subsubsection{6. Make statistical
conclusion}\label{make-statistical-conclusion}}

\begin{quote}
\textbf{\emph{Question:}}: Based on the P-value, do we accept or reject
the null hypothesis?\\
\textbf{\emph{Answer:}} \textless{} write your answer here
\textgreater{}
\end{quote}

\hypertarget{write-a-scientific-biological-conclusion}{%
\subsubsection{7. Write a scientific (biological)
conclusion}\label{write-a-scientific-biological-conclusion}}

\begin{quote}
\textbf{\emph{Question:}}: Now write a scientific (biological)
conclusion based on the outcome in 6.\\
\textbf{\emph{Answer:}} \textless{} write your answer here
\textgreater{}
\end{quote}

\hypertarget{exercise-2-stinging-trees-individual-or-in-pairs}{%
\section{Exercise 2: Stinging trees (individual or in
pairs)}\label{exercise-2-stinging-trees-individual-or-in-pairs}}

Data file: Stinging.csv

A forest ecologist, studying regeneration of rainforest communities in
gaps caused by large trees falling during storms, read that stinging
tree, \textbf{Dendrocnide excelsa}, seedlings will grow 1.5m/year in
direct sunlight such as gaps. In the gaps in her study plot, she
identified 9 specimens of this species and measure them in 1998 and
again 1 year later.

Does her data support the published contention that seedlings of this
species will average 1.5m of growth per year in direct sunlight? Also,
calculate a 95\% CI for the true mean. Analyse the data in R. Due to the
small sample size we have to assume the data is normal.

Work through the steps below individually or in pairs. Add more code
chunks if required (click insert -\textgreater{} R on above toolbar)

\begin{quote}
\begin{enumerate}
\def\labelenumi{\arabic{enumi}.}
\tightlist
\item
  Choose level of significance (α)\\
  Answer:
\end{enumerate}
\end{quote}

\begin{center}\rule{0.5\linewidth}{0.5pt}\end{center}

\begin{quote}
\begin{enumerate}
\def\labelenumi{\arabic{enumi}.}
\setcounter{enumi}{1}
\tightlist
\item
  Write null and alternate hypotheses\\
  H\textsubscript{0}:\\
  H\textsubscript{1}:
\end{enumerate}
\end{quote}

\begin{center}\rule{0.5\linewidth}{0.5pt}\end{center}

\begin{quote}
\begin{enumerate}
\def\labelenumi{\arabic{enumi}.}
\setcounter{enumi}{2}
\tightlist
\item
  Check assumptions (normal)
\end{enumerate}
\end{quote}

Read in the data:

\begin{Shaded}
\begin{Highlighting}[]
\CommentTok{\#library(readxl)}
\CommentTok{\#sting \textless{}{-} read\_excel("data/ENVX1002\_Data5.xlsx", sheet = "Stinging")}
\CommentTok{\#sting}
\end{Highlighting}
\end{Shaded}

Plot your data:

\begin{Shaded}
\begin{Highlighting}[]
\CommentTok{\# write your R code here}
\end{Highlighting}
\end{Shaded}

Normality tests:

\begin{Shaded}
\begin{Highlighting}[]
\CommentTok{\# write your R code here}
\end{Highlighting}
\end{Shaded}

\begin{quote}
\textbf{\emph{Question:}} Are data are normally distributed? Explain
your answer.\\
\textbf{\emph{Answer:}} \textless{} write your answer here
\textgreater{}
\end{quote}

\begin{center}\rule{0.5\linewidth}{0.5pt}\end{center}

\begin{quote}
\begin{enumerate}
\def\labelenumi{\arabic{enumi}.}
\setcounter{enumi}{3}
\tightlist
\item
  Calculate test statistic and\\
\item
  Obtain P-value or critical value
\end{enumerate}
\end{quote}

\begin{Shaded}
\begin{Highlighting}[]
\CommentTok{\# write your R code here}
\end{Highlighting}
\end{Shaded}

\begin{quote}
\begin{enumerate}
\def\labelenumi{\arabic{enumi}.}
\setcounter{enumi}{5}
\tightlist
\item
  Make statistical conclusion\\
  Answer:
\end{enumerate}
\end{quote}

\begin{center}\rule{0.5\linewidth}{0.5pt}\end{center}

\begin{quote}
\begin{enumerate}
\def\labelenumi{\arabic{enumi}.}
\setcounter{enumi}{6}
\tightlist
\item
  Write a scientific (biological) conclusion\\
  Answer:
\end{enumerate}
\end{quote}

\begin{center}\rule{0.5\linewidth}{0.5pt}\end{center}

\hypertarget{check-you-answers-with-teaching-staff}{%
\subsubsection{Check you answers with teaching
staff}\label{check-you-answers-with-teaching-staff}}

\hypertarget{thanks}{%
\subsection{Thanks!}\label{thanks}}

\hypertarget{attribution}{%
\subsubsection{Attribution}\label{attribution}}

This lab was developed using resources that are available under a
\href{http://creativecommons.org/licenses/by/4.0/}{Creative Commons
Attribution 4.0 International license}, made available on the
\href{https://github.com/usyd-soles-edu/}{SOLES Open Educational
Resources repository}.



\end{document}
